\section{Conclusion}

% We present {\videojam}, a new approach designed to meet the challenges of video analytics applications that integrate heterogeneous camera sources, \ie, both fixed and mobile. {\videojam} responds to scenarios incurring high load variability (such as mobile cameras) by integrating short term load prediction and performing load balancing at function level. Further, the system adapts to varying deployment configurations, not requiring any hard reboots to compensate for them. Thanks to its design, {\videojam} reduces response times by 2.91$\times$ lower response time, while reducing video data loss by more than 4.64$\times$  and generating lower bandwidth overheads.

% In future work, we plan to tackle the need of accounting of additional constraints in the analytics pipeline. Currently, {\videojam} does not consider inter-function link bandwidths to determine load balancing policies. In heterogeneous network environments, where link speeds differ, this omission can have an impact on overall system efficiency. In addition, while {\videojam} works independently of the existing deployment configuration (\eg, number of replicas for each function), it does not compensate for scenarios where the load exceeds the existing processing capabilities (\eg, too many video sources to process). Furthermore, it could be interesting to treat neighborhood cases with functions that are not necessarily identical, but rather functions with identical or similar objectives but slightly different implementation (\eg, \acrshort{yolo} and \acrshort{ssd}~\cite{liu2016ssd} are similar). {\videojam}, by default, can handle this heterogeneity, although they are treated as identical functions with different performances. However, these functions have many more differences, for example in terms of accuracy, size, inference time, \etc, and therefore raise more challenges.\\
% Future work will address these limitations, with the aim of improving resource utilization and exploring more adaptive deployment strategies.

We present {\videojam}, a new approach designed to meet the challenges of video analytics applications that integrate heterogeneous camera sources, i.e., both fixed and mobile. {\videojam} responds to scenarios incurring high load variability—particularly from mobile cameras—by integrating short-term load prediction and performing load balancing at the function level. The system adapts to varying deployment configurations without requiring hard reboots, enabling resilient and low-latency performance across dynamic environments. Empirical evaluations show that {\videojam} achieves 2.91× lower response times, reduces video data loss by over 4.64×, and minimizes bandwidth overhead.

While {\videojam} effectively balances computational workloads across distributed analytics pipelines, it assumes that the underlying infrastructure can absorb redistributed tasks without significant contention. In reality, many deployment environments—whether edge devices, on-premise clusters, or hybrid setups—operate under tight resource constraints. In such scenarios, multiple deep learning models often share the same hardware, leading to complex interference patterns that degrade inference performance. These challenges are especially pronounced when scaling deployments or colocating diverse models on shared GPUs, where traditional scheduling and placement strategies fall short.

To address this complementary dimension of resource management, our next contribution, {\roomie}, focuses on optimizing the cohabitation of DNN models in constrained environments. By profiling low-level GPU kernel execution behavior, {\roomie} enables intelligent orchestration of inference tasks, ensuring efficient model placement and predictable performance across a wide range of deployment contexts.