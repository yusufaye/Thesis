\section{Conclusion}\label{sec:conclusion}

We present VideoJam, a new approach designed to meet the challenges of video analytics applications that integrate heterogeneous camera sources, \ie, both fixed and mobile. VideoJam responds to scenarios incurring high load variability (such as mobile cameras) by integrating short term load prediction and performing load balancing at function level. Further, the system adapts to varying deployment configurations, not requiring any hard reboots to compensate for them. Thanks to its design, VideoJam reduces response times by 2.91$\times$ lower response time, while reducing video data loss by more than 4.64$\times$  and generating lower bandwidth overheads.

In future work, we plan to tackle the need of accounting of additional constraints in the analytics pipeline. Currently, VideoJam does not consider inter-function link bandwidths to determine load balancing policies. In heterogeneous network environments, where link speeds differ, this omission can have an impact on overall system efficiency. In addition, while VideoJam works independently of the existing deployment configuration (\eg, number of replicas for each function), it does not compensate for scenarios where the load exceeds the existing processing capabilities (\eg, too many video sources to process). Furthermore, it could be interesting to treat neighborhood cases with functions that are not necessarily identical, but rather functions with identical or similar objectives but slightly different implementation (\eg, YOLOv5 and SSD~\cite{liu2016ssd} are similar).~VideoJam, by default, can handle this heterogeneity, although they are treated as identical functions with different performances. However, these functions have many more differences, for example in terms of accuracy, size, inference time, \etc, and therefore raise more challenges.\\
Future work will address these limitations, with the aim of improving resource utilization and exploring more adaptive deployment strategies.