\paragraph{The Evolution of Edge Computing Architectures}

Edge computing represents a paradigm shift in distributed systems, wherein computational tasks and data storage are relocated closer to the data source. This architectural transformation addresses several limitations inherent in centralized cloud infrastructures, notably by reducing latency, minimizing bandwidth consumption, enhancing data security, and enabling real-time responsiveness. Given that edge devices—such as sensors, cameras, and Internet of Things (IoT) nodes—typically operate under constrained computational and energy resources, a variety of architectural models have emerged. These include hybrid edge-cloud systems, which balance local and remote processing, as well as fully autonomous edge analytics platforms designed for domain-specific applications, particularly within industrial IoT environments.

\paragraph{GPU Integration and Computational Enhancement.}
The incorporation of Graphics Processing Units (GPUs) into edge computing frameworks has significantly expanded the computational capabilities of edge systems. GPUs are inherently suited for parallel processing, rendering them ideal for compute-intensive tasks such as machine learning, computer vision, and large-scale data analytics. Their deployment at the edge yields notable improvements in performance, efficiency, and scalability. As a result, GPUs have become instrumental in enabling advanced edge applications, including autonomous vehicles, smart surveillance systems, immersive augmented and virtual reality (AR/VR) experiences, and responsive healthcare diagnostics. These applications benefit from the edge's capacity to process data locally and in real time, thereby reducing latency and enhancing decision-making accuracy. In industrial IoT contexts, for example, edge computing facilitates real-time monitoring and control of equipment, leading to increased operational efficiency and reduced downtime.

\paragraph{Video Analytics as a Critical Edge Application.}
Among the diverse applications of edge computing, video analytics emerges as a particularly demanding and impactful use case. In domains such as smart cities, autonomous driving, and public safety, edge-based video analytics enables the timely interpretation of visual data. A typical video analytics pipeline comprises several stages: data ingestion, preprocessing, feature extraction, model inference, and post-processing. Each stage imposes distinct computational and memory requirements, which are often challenging to satisfy within the limited capabilities of edge devices.

\paragraph{Resource Constraints and Architectural Challenges.}
The computational limitations of edge devices—manifested in restricted processing power, limited storage capacity, and constrained communication bandwidth, pose significant challenges to the deployment of complex video analytics workloads. These constraints necessitate the development of innovative strategies aimed at optimizing resource utilization while preserving the real-time performance and locality benefits of edge computing.

\paragraph{Optimization Strategies and Emerging Solutions.}
To address these challenges, researchers have proposed a spectrum of solutions designed to enhance performance without compromising the advantages of edge deployment. Techniques such as model compression and pruning have been employed to reduce the computational footprint of deep learning algorithms, thereby enabling their execution on resource-constrained devices. Distributed processing frameworks facilitate the partitioning and collaborative execution of workloads across multiple edge nodes, improving scalability and fault tolerance. Furthermore, adaptive scheduling algorithms and energy-aware resource management strategies have been developed to dynamically allocate tasks and balance performance with sustainability. Collectively, these approaches contribute to the realization of robust, efficient, and scalable edge computing systems capable of supporting increasingly complex and latency-sensitive applications.